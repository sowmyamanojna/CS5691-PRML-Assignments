\documentclass[11pt,a4paper]{article}
\usepackage{amsmath}
\usepackage{amsfonts}
\usepackage{amssymb}
\usepackage{makeidx}
\usepackage{graphicx}
\usepackage{wrapfig}
\usepackage{enumerate}
\usepackage{pdfpages}
\usepackage{tocloft}
\usepackage{setspace}
\usepackage{mathtools}
\usepackage{hyperref}
\definecolor{linkcolour}{rgb}{0,0.2,0.6} % Link color
\hypersetup{colorlinks,breaklinks,urlcolor=linkcolour,linkcolor=linkcolour}

\usepackage[left=2cm,right=2cm,top=1.5cm,bottom=1.5cm]{geometry}

\usepackage{xcolor}

\usepackage{color,soul}
\usepackage{fontspec}
\setmainfont{Cambria}

\usepackage{caption}
\captionsetup[figure]{font=small, labelfont={bf}}
\captionsetup[table]{font=small, labelfont={bf}}

\usepackage{float}
\usepackage{multirow}
\usepackage{longtable}

\usepackage[nottoc]{tocbibind}
\usepackage[framed,numbered,autolinebreaks,useliterate]{mcode}

\newcommand{\spa}{\vspace{1.25em}}
\newcommand{\noi}{\noindent}
\def\dul#1{\underline{\underline{#1}}}
\def\cpt#1#2{{\begin{center}\small\textbf{\textcolor{blue}{Figure #1:}} #2\end{center}}}
\def\tt#1{\texttt{#1}}

% for dots in the content
\usepackage{tocloft}
\renewcommand{\cftsecleader}{\cftdotfill{\cftdotsep}}

\begin{document}
    \begin{titlepage} 
        \begin{center}
        \large{ASSIGNMENT 2}\\
        \vspace{2em}
        \large {CS5691 Pattern Recognition and Machine Learning}
        \vspace{3em}
        
        \rule{0.9\linewidth}{0.5mm} \\[0.4cm]
        {\Large{\bfseries{CS5691 Assignment Code 2}}} \\
        \rule{0.9\linewidth}{0.5mm} \\[3 em]    
        
        Team Members: \\
        \vspace{0.5em}
        \def\arraystretch{1.25}
\begin{tabular}{c l}
	\hline
	BE17B007 & N Sowmya Manojna \\
	PH17B010 & Thakkar Riya Anandbhai \\
	PH17B011 & Chaithanya Krishna Moorthy \\
	\hline
\end{tabular}

        \vspace{1em}

        Indian Institute of Technology, Madras\\    
        
        \vspace{5em}    
        
            \includegraphics[scale = 0.09]{images/iitmlogo.png}
        \end{center}
    \end{titlepage}

{\hypersetup{linkcolor=black}
 \tableofcontents}
\break

\noi
\textcolor{blue}{All codes excluding the modules are converted to \tt{.py} files from IPython Notebooks}
\section{Dataset 1A}
The code written for analyzing Dataset 1A is as follows:\vspace{-1em}
\lstinputlisting[language=Python]{codes/dataset1a.py}

\section{Dataset 1B}
\subsection{Bayes Classification, GMM, Full Covariance}
The GMM full covariance model code is as follows: \vspace{-1em}
\lstinputlisting[language=Python]{codes/1b-full-covariance.py}

\noi
The GMM class module is as follows:\vspace{-1em}
\lstinputlisting[language=Python]{codes/gmm.py}

\subsection{Bayes Classification, GMM, Diagonal Covariance}
The GMM diagonal covariance model code is as follows: \vspace{-1em}
\lstinputlisting[language=Python]{codes/Assignment2.py}

\noi
The accuracy module is as follows:\vspace{-1em}
\lstinputlisting[language=Python]{codes/accuracy.py}

\subsection{Bayes Classification, KNN}
\lstinputlisting[language=Python]{codes/dataset1b.py}

\section{Dataset 2A}
\subsection{Bayes Classification, GMM, Full Covariance}
The GMM full covariance model code is as follows: \vspace{-1em}
\lstinputlisting[language=Python]{codes/2a-full-covariance.py}

\noi
The GMM class module is as follows:\vspace{-1em}
\lstinputlisting[language=Python]{codes/gmm.py}

\noi
The utils script is as follows:\vspace{-1em}
\lstinputlisting[language=Python]{codes/utils.py}


\subsection{Bayes Classification, GMM, Diagonal Covariance}
The GMM diagonal covariance model code is as follows: \vspace{-1em}
\lstinputlisting[language=Python]{codes/Assignment2-2a.py}

\section{Dataset 2B}
The GMM class module is as follows:\vspace{-1em}
\lstinputlisting[language=Python]{codes/gmm.py}

The code used is as follows:\vspace{-1em}
\lstinputlisting[language=Python]{codes/2b-full-covariance.py}

\end{document}
