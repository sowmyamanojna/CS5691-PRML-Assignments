\documentclass[11pt,a4paper]{article}
\usepackage{amsmath}
\usepackage{amsfonts}
\usepackage{amssymb}
\usepackage{makeidx}
\usepackage{graphicx}
\usepackage{wrapfig}
\usepackage{enumerate}
\usepackage{pdfpages}
\usepackage{tocloft}
\usepackage{setspace}
\usepackage{mathtools}
\usepackage{hyperref}
\definecolor{linkcolour}{rgb}{0,0.2,0.6} % Link color
\hypersetup{colorlinks,breaklinks,urlcolor=linkcolour,linkcolor=linkcolour}

\usepackage[left=2cm,right=2cm,top=1.5cm,bottom=1.5cm]{geometry}

\usepackage{xcolor}

\usepackage{color,soul}
\usepackage{fontspec}
\setmainfont{Cambria}

\usepackage{caption}
\captionsetup[figure]{font=small, labelfont={bf}}
\captionsetup[table]{font=small, labelfont={bf}}

\usepackage{float}
\usepackage{multirow}
\usepackage{longtable}

\usepackage[nottoc]{tocbibind}

\newcommand{\spa}{\vspace{1.25em}}
\newcommand{\noi}{\noindent}
\def\dul#1{\underline{\underline{#1}}}
\def\cpt#1#2{{\begin{center}\small\textbf{\textcolor{blue}{Figure #1:}} #2\end{center}}}
% \def\tt#1{\texttt{#1}}
\def\colortt#1{\textcolor{blue}{\texttt{#1}}}

% for dots in the content
\usepackage{tocloft}
\renewcommand{\cftsecleader}{\cftdotfill{\cftdotsep}}

\usepackage{titlesec}
\setcounter{secnumdepth}{4}
\setcounter{tocdepth}{4}
\titleformat{\paragraph}
{\normalfont\normalsize\bfseries}{\theparagraph}{1em}{}
\titlespacing*{\paragraph}
{0pt}{3.25ex plus 1ex minus .2ex}{1.5ex plus .2ex}

\begin{document}
	\begin{titlepage} 
		\begin{center}
		\large{ASSIGNMENT 3}\\
		\vspace{2em}
		\large {CS5691 Pattern Recognition and Machine Learning}
		\vspace{3em}
		
		\rule{0.9\linewidth}{0.5mm} \\[0.4cm]
	    {\Large{\bfseries{CS5691 Assignment 3}}} \\
	    \rule{0.9\linewidth}{0.5mm} \\[3 em]	
	    
	    Team Members: \\
	    \vspace{0.5em}
	   	\def\arraystretch{1.25}
\begin{tabular}{c l}
	\hline
	BE17B007 & N Sowmya Manojna \\
	PH17B010 & Thakkar Riya Anandbhai \\
	PH17B011 & Chaithanya Krishna Moorthy \\
	\hline
\end{tabular}

		\vspace{1em}

		Indian Institute of Technology, Madras\\    
		
		\vspace{5em}    
	    
	    	\includegraphics[scale = 0.09]{images/iitmlogo.png}
		\end{center}
	\end{titlepage}

{\hypersetup{linkcolor=black}
 \tableofcontents}
\break


%%%%%%%%%%%%%%%%%%%%%%%%%%%%%%%%%%%%%%%%%%%%%%%
%%%%%%%%%%%%%%%%%%%%%%%%%%%%%%%%%%%%%%%%%%%%%%%
\section{Dataset 1A}
%%%%%%%%%%%%%%%%%%%%%%%%%%%%%%%%%%%%%%%%%%%%%%%
%%%%%%%%%%%%%%%%%%%%%%%%%%%%%%%%%%%%%%%%%%%%%%%
\subsection{Perceptron}


%%%%%%%%%%%%%%%%%%%%%%%%%%%%%%%%%%%%%%%%%%%%%%%
\subsection{MLFFNN}
\subsubsection{Classification Accuracies}
The classification accuracies on the training and validation datasets are as follows:
\def\arraystretch{1.25}
\begin{table}[H]
\centering
{\small
\begin{tabular}{l l l l l l l c}
\hline
\hline
\textbf{\# Neurons} & \textbf{Activation} & \textbf{Solver} & \textbf{Batch Size} & \textbf{$\alpha$} & \textbf{Learning Rate} & \textbf{Accuracy} & \textbf{Validation Accuracy} \\
\hline
\hline
5 & tanh & lbfgs & 200 & 0.0001 & adaptive & 100.0 & 100.0 \\
5 & tanh & lbfgs & 200 & 0.0001 & constant & 100.0 & 100.0 \\
5 & tanh & lbfgs & 200 & 0.0 & invscaling & 100.0 & 100.0 \\
5 & tanh & lbfgs & 200 & 0.0 & adaptive & 100.0 & 100.0 \\
5 & tanh & lbfgs & 200 & 0.0 & constant & 100.0 & 100.0 \\
5 & tanh & lbfgs & 100 & 0.0 & adaptive & 100.0 & 100.0 \\
5 & tanh & lbfgs & 100 & 0.0001 & invscaling & 100.0 & 100.0 \\
5 & relu & lbfgs & 200 & 0.0 & constant & 100.0 & 100.0 \\
5 & relu & lbfgs & 100 & 0.0001 & invscaling & 100.0 & 100.0 \\
5 & relu & lbfgs & 200 & 0.0 & adaptive & 100.0 & 100.0 \\
\hline
\end{tabular}
\caption{Best 10 Train and Validation Accuracies obtained after performing a \colortt{GridSearch} on 432 parameter combinations.}
}
\end{table}

\subsubsection{Best Model}
The parameter combination were additionally sorted based on minimum fitting time (least fitting time - first) and the model that gave the best accuracy the fastest was chosen. Hence the best parameter combination chosen is:
\begin{itemize}
    \itemsep0em
    \item hidden\_layer\_sizes: 5
    \item activation: tanh
    \item solver: lbfgs
    \item batch\_size: 200
    \item alpha: 0.0001
    \item learning\_rate: adaptive
\end{itemize}

\noi
The classification accuracy of the best model on the testing data is: $100\%$. The confusion matrices obtained are as follows:
\begin{figure}[H]
    \centering
    \includegraphics[scale=0.35]{images/1A_MLFFNN_train_confmat.png}
    \includegraphics[scale=0.35]{images/1A_MLFFNN_val_confmat.png}
    \caption{Training and Validation confusion matrices obtained for the best parameter combination, on the left and right respectively.}
\end{figure}

\begin{figure}[H]
    \centering
    \includegraphics[scale=0.4]{images/1A_MLFFNN_test_confmat.png}
    \caption{Testing confusion matrices obtained for the best parameter combination.}
\end{figure}

\subsubsection{Decision Region}
The decision region plots obtained is as follows:
\begin{figure}[H]
    \centering
    \includegraphics[scale=0.6]{images/1A_MLFFNN_Decision_Plot.png}
    \caption{Decision Region Plot obtained for the best parameter combination.}
\end{figure}

%%%%%%%%%%%%%%%%%%%%%%%%%%%%%%%%%%%%%%%%%%%%%%%
\subsection{Linear SVM}

\break
%%%%%%%%%%%%%%%%%%%%%%%%%%%%%%%%%%%%%%%%%%%%%%%
%%%%%%%%%%%%%%%%%%%%%%%%%%%%%%%%%%%%%%%%%%%%%%%
\section{Dataset 1B}
%%%%%%%%%%%%%%%%%%%%%%%%%%%%%%%%%%%%%%%%%%%%%%%
%%%%%%%%%%%%%%%%%%%%%%%%%%%%%%%%%%%%%%%%%%%%%%%
\subsection{MLFFNN}
\subsubsection{Classification Accuracies}
The classification accuracies on the training and validation datasets are as follows:
\def\arraystretch{1.25}
\begin{table}[H]
\centering
{\small
\begin{tabular}{l l l l l l c c}
\hline
\hline
\textbf{\# Neurons} & \textbf{Activation} & \textbf{Batch Size} & \textbf{Early Stopping} & \textbf{Learning Rate} & \textbf{$\alpha$} & \textbf{Accuracy} & \textbf{Validation Accuracy} \\
\hline
\hline
(8, 8) & relu & 50 & False & adaptive & 0.01 & 99.33 & 98.41  \\
(8, 8) & relu & 50 & False & constant & 0.001 & 99.33 & 98.41  \\
(8, 8) & relu & 50 & False & invscaling & 0.01 & 99.33 & 98.41  \\
(8, 8) & relu & 50 & False & adaptive & 0.001 & 99.33 & 98.41  \\
(8, 8) & relu & 50 & False & invscaling & 0.001 & 99.33 & 98.41  \\
(8, 8) & relu & 50 & False & constant & 0.01 & 99.33 & 98.41  \\
(10, 10) & relu & 50 & False & adaptive & 0.01 & 99.0 & 98.41  \\
(10, 10) & relu & 50 & False & constant & 0.01 & 99.0 & 98.41  \\
(10, 10) & relu & 50 & False & invscaling & 0.01 & 99.0 & 98.41  \\
(10, 10) & relu & 50 & False & constant & 0.001 & 99.0 & 96.82  \\
\hline
\end{tabular}
\caption{Best 10 Train and Validation Accuracies obtained after performing a \colortt{GridSearch} on 432 parameter combinations.}
}
\end{table}

\subsubsection{Best Model}
The parameter combination were additionally sorted based on minimum fitting time (least fitting time - first) and the model that gave the best accuracy the fastest was chosen. Hence the best parameter combination chosen is:
\begin{itemize}
    \itemsep0em
    \item hidden\_layer\_sizes: (8, 8)
    \item activation: relu
    \item batch\_size: 50
    \item early\_stopping: False
    \item learning\_rate: adaptive
    \item alpha: 0.01
\end{itemize}

\noi
The classification accuracy of the best model on the testing data is: $96.296\%$. The confusion matrices obtained are as follows:
\begin{figure}[H]
    \centering
    \includegraphics[scale=0.5]{images/1B_MLFFNN_train_confmat.png}
    \includegraphics[scale=0.5]{images/1B_MLFFNN_val_confmat.png}
    \caption{Training and Validation confusion matrices obtained for the best parameter combination, on the left and right respectively.}
\end{figure}

\begin{figure}[H]
    \centering
    \includegraphics[scale=0.4]{images/1B_MLFFNN_test_confmat.png}
    \caption{Testing confusion matrices obtained for the best parameter combination.}
\end{figure}

\subsubsection{Decision Region}
The decision region plots obtained is as follows:
\begin{figure}[H]
    \centering
    \includegraphics[scale=0.6]{images/1B_MLFFNN_Decision_Plot.png}
    \caption{Decision Region Plot obtained for the best parameter combination.}
\end{figure}

\subsubsection{Surface Plots}
The neuron-wise surface plots obtained for the hidden and output layers is as follows:
\paragraph{Hidden Layer 1, Node 1}
\begin{figure}[H]
    \centering
    \includegraphics[scale=0.4]{images/1B_MLFFNN_E1_HL1_N1.png}
    \includegraphics[scale=0.4]{images/1B_MLFFNN_E5_HL1_N1.png}
    \includegraphics[scale=0.4]{images/1B_MLFFNN_E20_HL1_N1.png}
    \includegraphics[scale=0.4]{images/1B_MLFFNN_E100_HL1_N1.png}
    \includegraphics[scale=0.4]{images/1B_MLFFNN_conv_HL1_N1.png}
    \caption{Surface Plots obtained for Hidden Layer 1, Neuron 1, across epochs.}
    \label{HL1N1}
\end{figure}

\paragraph{Hidden Layer 1, Node 2}
\begin{figure}[H]
    \centering
    \includegraphics[scale=0.4]{images/1B_MLFFNN_E1_HL1_N2.png}
    \includegraphics[scale=0.4]{images/1B_MLFFNN_E5_HL1_N2.png}
    \includegraphics[scale=0.4]{images/1B_MLFFNN_E20_HL1_N2.png}
    \includegraphics[scale=0.4]{images/1B_MLFFNN_E100_HL1_N2.png}
    \includegraphics[scale=0.4]{images/1B_MLFFNN_conv_HL1_N2.png}
    \caption{Surface Plots obtained for Hidden Layer 1, Neuron 2, across epochs.}
\end{figure}

\paragraph{Hidden Layer 1, Node 3}
\begin{figure}[H]
    \centering
    \includegraphics[scale=0.4]{images/1B_MLFFNN_E1_HL1_N3.png}
    \includegraphics[scale=0.4]{images/1B_MLFFNN_E5_HL1_N3.png}
    \includegraphics[scale=0.4]{images/1B_MLFFNN_E20_HL1_N3.png}
    \includegraphics[scale=0.4]{images/1B_MLFFNN_E100_HL1_N3.png}
    \includegraphics[scale=0.4]{images/1B_MLFFNN_conv_HL1_N3.png}
    \caption{Surface Plots obtained for Hidden Layer 1, Neuron 3, across epochs.}
\end{figure}

\paragraph{Hidden Layer 1, Node 4}
\begin{figure}[H]
    \centering
    \includegraphics[scale=0.4]{images/1B_MLFFNN_E1_HL1_N4.png}
    \includegraphics[scale=0.4]{images/1B_MLFFNN_E5_HL1_N4.png}
    \includegraphics[scale=0.4]{images/1B_MLFFNN_E20_HL1_N4.png}
    \includegraphics[scale=0.4]{images/1B_MLFFNN_E100_HL1_N4.png}
    \includegraphics[scale=0.4]{images/1B_MLFFNN_conv_HL1_N4.png}
    \caption{Surface Plots obtained for Hidden Layer 1, Neuron 4, across epochs.}
\end{figure}

\paragraph{Hidden Layer 1, Node 5}
\begin{figure}[H]
    \centering
    \includegraphics[scale=0.4]{images/1B_MLFFNN_E1_HL1_N5.png}
    \includegraphics[scale=0.4]{images/1B_MLFFNN_E5_HL1_N5.png}
    \includegraphics[scale=0.4]{images/1B_MLFFNN_E20_HL1_N5.png}
    \includegraphics[scale=0.4]{images/1B_MLFFNN_E100_HL1_N5.png}
    \includegraphics[scale=0.4]{images/1B_MLFFNN_conv_HL1_N5.png}
    \caption{Surface Plots obtained for Hidden Layer 1, Neuron 5, across epochs.}
\end{figure}

\paragraph{Hidden Layer 1, Node 6}
\begin{figure}[H]
    \centering
    \includegraphics[scale=0.4]{images/1B_MLFFNN_E1_HL1_N6.png}
    \includegraphics[scale=0.4]{images/1B_MLFFNN_E5_HL1_N6.png}
    \includegraphics[scale=0.4]{images/1B_MLFFNN_E20_HL1_N6.png}
    \includegraphics[scale=0.4]{images/1B_MLFFNN_E100_HL1_N6.png}
    \includegraphics[scale=0.4]{images/1B_MLFFNN_conv_HL1_N6.png}
    \caption{Surface Plots obtained for Hidden Layer 1, Neuron 6, across epochs.}
\end{figure}

\paragraph{Hidden Layer 1, Node 7}
\begin{figure}[H]
    \centering
    \includegraphics[scale=0.4]{images/1B_MLFFNN_E1_HL1_N7.png}
    \includegraphics[scale=0.4]{images/1B_MLFFNN_E5_HL1_N7.png}
    \includegraphics[scale=0.4]{images/1B_MLFFNN_E20_HL1_N7.png}
    \includegraphics[scale=0.4]{images/1B_MLFFNN_E100_HL1_N7.png}
    \includegraphics[scale=0.4]{images/1B_MLFFNN_conv_HL1_N7.png}
    \caption{Surface Plots obtained for Hidden Layer 1, Neuron 7, across epochs.}
\end{figure}

\paragraph{Hidden Layer 1, Node 8}
\begin{figure}[H]
    \centering
    \includegraphics[scale=0.4]{images/1B_MLFFNN_E1_HL1_N8.png}
    \includegraphics[scale=0.4]{images/1B_MLFFNN_E5_HL1_N8.png}
    \includegraphics[scale=0.4]{images/1B_MLFFNN_E20_HL1_N8.png}
    \includegraphics[scale=0.4]{images/1B_MLFFNN_E100_HL1_N8.png}
    \includegraphics[scale=0.4]{images/1B_MLFFNN_conv_HL1_N8.png}
    \caption{Surface Plots obtained for Hidden Layer 1, Neuron 8, across epochs.}
\end{figure}

\paragraph{Hidden Layer 2, Node 1}
\begin{figure}[H]
    \centering
    \includegraphics[scale=0.4]{images/1B_MLFFNN_E1_HL2_N1.png}
    \includegraphics[scale=0.4]{images/1B_MLFFNN_E5_HL2_N1.png}
    \includegraphics[scale=0.4]{images/1B_MLFFNN_E20_HL2_N1.png}
    \includegraphics[scale=0.4]{images/1B_MLFFNN_E100_HL2_N1.png}
    \includegraphics[scale=0.4]{images/1B_MLFFNN_conv_HL2_N1.png}
    \caption{Surface Plots obtained for Hidden Layer 2, Neuron 1, across epochs.}
\end{figure}

\paragraph{Hidden Layer 2, Node 2}
\begin{figure}[H]
    \centering
    \includegraphics[scale=0.4]{images/1B_MLFFNN_E1_HL2_N2.png}
    \includegraphics[scale=0.4]{images/1B_MLFFNN_E5_HL2_N2.png}
    \includegraphics[scale=0.4]{images/1B_MLFFNN_E20_HL2_N2.png}
    \includegraphics[scale=0.4]{images/1B_MLFFNN_E100_HL2_N2.png}
    \includegraphics[scale=0.4]{images/1B_MLFFNN_conv_HL2_N2.png}
    \caption{Surface Plots obtained for Hidden Layer 2, Neuron 2, across epochs.}
\end{figure}

\paragraph{Hidden Layer 2, Node 3}
\begin{figure}[H]
    \centering
    \includegraphics[scale=0.4]{images/1B_MLFFNN_E1_HL2_N3.png}
    \includegraphics[scale=0.4]{images/1B_MLFFNN_E5_HL2_N3.png}
    \includegraphics[scale=0.4]{images/1B_MLFFNN_E20_HL2_N3.png}
    \includegraphics[scale=0.4]{images/1B_MLFFNN_E100_HL2_N3.png}
    \includegraphics[scale=0.4]{images/1B_MLFFNN_conv_HL2_N3.png}
    \caption{Surface Plots obtained for Hidden Layer 2, Neuron 3, across epochs.}
\end{figure}

\paragraph{Hidden Layer 2, Node 4}
\begin{figure}[H]
    \centering
    \includegraphics[scale=0.4]{images/1B_MLFFNN_E1_HL2_N4.png}
    \includegraphics[scale=0.4]{images/1B_MLFFNN_E5_HL2_N4.png}
    \includegraphics[scale=0.4]{images/1B_MLFFNN_E20_HL2_N4.png}
    \includegraphics[scale=0.4]{images/1B_MLFFNN_E100_HL2_N4.png}
    \includegraphics[scale=0.4]{images/1B_MLFFNN_conv_HL2_N4.png}
    \caption{Surface Plots obtained for Hidden Layer 2, Neuron 4, across epochs.}
\end{figure}

\paragraph{Hidden Layer 2, Node 5}
\begin{figure}[H]
    \centering
    \includegraphics[scale=0.4]{images/1B_MLFFNN_E1_HL2_N5.png}
    \includegraphics[scale=0.4]{images/1B_MLFFNN_E5_HL2_N5.png}
    \includegraphics[scale=0.4]{images/1B_MLFFNN_E20_HL2_N5.png}
    \includegraphics[scale=0.4]{images/1B_MLFFNN_E100_HL2_N5.png}
    \includegraphics[scale=0.4]{images/1B_MLFFNN_conv_HL2_N5.png}
    \caption{Surface Plots obtained for Hidden Layer 2, Neuron 5, across epochs.}
\end{figure}

\paragraph{Hidden Layer 2, Node 6}
\begin{figure}[H]
    \centering
    \includegraphics[scale=0.4]{images/1B_MLFFNN_E1_HL2_N6.png}
    \includegraphics[scale=0.4]{images/1B_MLFFNN_E5_HL2_N6.png}
    \includegraphics[scale=0.4]{images/1B_MLFFNN_E20_HL2_N6.png}
    \includegraphics[scale=0.4]{images/1B_MLFFNN_E100_HL2_N6.png}
    \includegraphics[scale=0.4]{images/1B_MLFFNN_conv_HL2_N6.png}
    \caption{Surface Plots obtained for Hidden Layer 2, Neuron 6, across epochs.}
\end{figure}

\paragraph{Hidden Layer 2, Node 7}
\begin{figure}[H]
    \centering
    \includegraphics[scale=0.4]{images/1B_MLFFNN_E1_HL2_N7.png}
    \includegraphics[scale=0.4]{images/1B_MLFFNN_E5_HL2_N7.png}
    \includegraphics[scale=0.4]{images/1B_MLFFNN_E20_HL2_N7.png}
    \includegraphics[scale=0.4]{images/1B_MLFFNN_E100_HL2_N7.png}
    \includegraphics[scale=0.4]{images/1B_MLFFNN_conv_HL2_N7.png}
    \caption{Surface Plots obtained for Hidden Layer 2, Neuron 7, across epochs.}
\end{figure}

\paragraph{Hidden Layer 2, Node 8}
\begin{figure}[H]
    \centering
    \includegraphics[scale=0.4]{images/1B_MLFFNN_E1_HL2_N8.png}
    \includegraphics[scale=0.4]{images/1B_MLFFNN_E5_HL2_N8.png}
    \includegraphics[scale=0.4]{images/1B_MLFFNN_E20_HL2_N8.png}
    \includegraphics[scale=0.4]{images/1B_MLFFNN_E100_HL2_N8.png}
    \includegraphics[scale=0.4]{images/1B_MLFFNN_conv_HL2_N8.png}
    \caption{Surface Plots obtained for Hidden Layer 2, Neuron 8, across epochs.}
\end{figure}

\paragraph{Output Layer, Node 1}
\begin{figure}[H]
    \centering
    \includegraphics[scale=0.4]{images/1B_MLFFNN_E1_OP_N1.png}
    \includegraphics[scale=0.4]{images/1B_MLFFNN_E5_OP_N1.png}
    \includegraphics[scale=0.4]{images/1B_MLFFNN_E20_OP_N1.png}
    \includegraphics[scale=0.4]{images/1B_MLFFNN_E100_OP_N1.png}
    \includegraphics[scale=0.4]{images/1B_MLFFNN_conv_OP_N1.png}
    \caption{Surface Plots obtained for Output Layer, Neuron 1, across epochs.}
\end{figure}

\paragraph{Output Layer, Node 2}
\begin{figure}[H]
    \centering
    \includegraphics[scale=0.4]{images/1B_MLFFNN_E1_OP_N2.png}
    \includegraphics[scale=0.4]{images/1B_MLFFNN_E5_OP_N2.png}
    \includegraphics[scale=0.4]{images/1B_MLFFNN_E20_OP_N2.png}
    \includegraphics[scale=0.4]{images/1B_MLFFNN_E100_OP_N2.png}
    \includegraphics[scale=0.4]{images/1B_MLFFNN_conv_OP_N2.png}
    \caption{Surface Plots obtained for Output Layer, Neuron 2, across epochs.}
\end{figure}

\paragraph{Output Layer, Node 3}
\begin{figure}[H]
    \centering
    \includegraphics[scale=0.4]{images/1B_MLFFNN_E1_OP_N3.png}
    \includegraphics[scale=0.4]{images/1B_MLFFNN_E5_OP_N3.png}
    \includegraphics[scale=0.4]{images/1B_MLFFNN_E20_OP_N3.png}
    \includegraphics[scale=0.4]{images/1B_MLFFNN_E100_OP_N3.png}
    \includegraphics[scale=0.4]{images/1B_MLFFNN_conv_OP_N3.png}
    \caption{Surface Plots obtained for Output Layer, Neuron 3, across epochs.}
    \label{OPN3}
\end{figure}

\noi
From \autoref{HL1N1}-\autoref{OPN3}, we observe the following:
\begin{itemize}
    \itemsep0em 
    \item First hidden layer surface plot is linear (activation function is \colortt{ReLU}).
    \item Non-Linearity is introduced in the surface plot of second hidden layer.
    \item The surface plot of the output neurons shows the selection cum localization of different classes in the latent space.
\end{itemize}

%%%%%%%%%%%%%%%%%%%%%%%%%%%%%%%%%%%%%%%%%%%%%%%
\subsection{Non-Linear SVM}

%%%%%%%%%%%%%%%%%%%%%%%%%%%%%%%%%%%%%%%%%%%%%%%
\break
%%%%%%%%%%%%%%%%%%%%%%%%%%%%%%%%%%%%%%%%%%%%%%%
\section{Dataset 2A}
%%%%%%%%%%%%%%%%%%%%%%%%%%%%%%%%%%%%%%%%%%%%%%%
%%%%%%%%%%%%%%%%%%%%%%%%%%%%%%%%%%%%%%%%%%%%%%%
\subsection{MLFFNN}
%%%%%%%%%%%%%%%%%%%%%%%%%%%%%%%%%%%%%%%%%%%%%%%
\subsection{Gaussian-kernel SVM}

\end{document}
